\documentclass[oneside,final,12pt]{extreport}
%should it be twoside or oneside??
\usepackage[utf8]{inputenc}
\usepackage[russianb]{babel}
\usepackage{vmargin}
\setpapersize{A4}
%   left, top, right, bottom, headers/footers(3), page number
\setmarginsrb{3cm}{2cm}{1.5cm}{2cm}{0pt}{0mm}{0pt}{13mm}
% \fussy
\sloppy
\usepackage{indentfirst}

\usepackage{graphicx}
\usepackage{wrapfig}

\usepackage{amsmath,amsfonts,amssymb,amsthm,mathtools}
\usepackage{siunitx}
\usepackage{physics}
\usepackage{hyperref}

\usepackage{placeins}

% \usepackage{cite} % multiref
\usepackage[
  backend=biber,
  style=numeric-comp,
  sorting=none,
]{biblatex}
\addbibresource{bibliography.bib}
\addbibresource{bib/springer_handbook.bib}
\addbibresource{bib/phys_chem_hydro_layers.bib}
% \bibliography{bibliography,bib/springer_handbook.bib,bib/phys_chem_hydro_layers.bib}

\usepackage[labelsep=endash]{caption}
\addto\captionsrussian{\renewcommand{\figurename}{Рисунок}}
\addto\captionsrussian{\renewcommand{\tablename}{Таблица}}

% \renewcommand{\thesection}{\arabic{section}} % remove 0. prefix
\newcommand\Section[1]{
  \FloatBarrier{}
  \refstepcounter{section}
  \section*{\centering
    \arabic{section}. #1}
  \addcontentsline{toc}{section}{%
    \arabic{section}. #1}
}
\newcommand\Subsection[1]{
  \FloatBarrier{}
  \refstepcounter{subsection}
  \subsection*{\centering
    \arabic{section}.\arabic{subsection}. #1}
  \addcontentsline{toc}{subsection}{%
    \arabic{section}.\arabic{subsection}. #1}
}

\newcommand\const[0]{\text{const}}
\newcommand\bcdot[0]{\boldsymbol\cdot}
\newcommand\bnabla[0]{\boldsymbol\nabla}



\begin{document}

% \begin{titlepage}

% Титульник.

% \end{titlepage}
\setcounter{page}{2}

\section*{Аннотация.}
% Цели и задачи. Результаты. Рекомендации, предложенные на основании работы.
Целью данной работы является освоение методов математического моделирования
применительно к массопереносу в микрофлюидных системах.
% TODO переместить "с помощью COMOL" в другое место, куда???
Для достижения этой цели было с помощью пакета программ COMSOL Multiphysics\texttrademark{}
смоделировано специфическое связывание аналита из анализируемого раствора
с модифицированной поверхностью:
\begin{itemize}
  \item микроканала, по которому движется аналит,%, через который протекает раствор,
  % \item модифицированной мембраной, через которую протекает раствор
  \item флуоресцентных микросфер, находящихся в анализируемом растворе.

\end{itemize}


С помощью данных моделей удалось показать справедливость
представлений о вязкостном и диффузионном граничных слоях,
как об одномерных приближениях в задаче
об адсорбции на стенке канала.
%   % \item обнаружить влияние учёта массопереноса в области пространства
%   %   за пределами диффузионного слоя в задаче об адсорбции на поверхность канала,
%   \item показать, что одномерный вариант задачи об адсорбции на поверхности канала
%     пригоден для моделирования отмывки от неспецифически связавшегося с поверхностью вещества.


\tableofcontents

\clearpage

\section*{Обозначения и сокращения}
\begin{description}
  \item[УрЧП:] уравнение в частных производных, уравнение математической физики.
%   \item[Диагностическе системы:]
  \item[(Не)специфическая адсорбция:] (не)специфическое связывание
    молекулы из раствора с молекулой на поверхности или с самой поверхностью,
    НЕ имеется в виду разница между адсорбцией, обусловленной
    электростатическим притяжением иона к поверхности
    и обусловленной силами Ван-дер-Ваальса.
  \item[Уравнение:] под уравнением зачастую имеется в виду система уравнений.
  \item[ППР:] поверхностный плазмонный резонанс.
  \item[ОФК:] одномерный фотонный кристалл.
  \item[КМ (QCM):] кварцевые микровесы (quartz crystal microbalance).
  \item[ЛНЧ (LOC):] лаборатория на чипе (lab-on-a-chip).
  \item[СМТА ($\mu$TAS):] система микрототального анализа (micro total analysis system).

\end{description}

\Section{Введение}
% Новизна, актуальность, цели и задачи.
% TODO что лучше?
Микрофлюидика --- научно-инженерная область, посвящённая поведению
% Микрофлюидика --- область науки и инженерии, посвящённая поведению
малых объёмов жидкости при малых потоках.
Микрофлюидика применяестя в биологии, медицине и нанотехнологиях.

В связи с малыми размерами микрофлюидных систем их поведение не интуитивно,
а наблюдения за состоянием и поведением таких систем затруднены.
Из-за этого в микрофлюидной практике
для создания систем с предсказуемыми свойствами,
а также для интерпретации получаемых с их помощью данных
применяется математическое моделирование.
% Для этих целей применим пакет программ COMSOL Multiphysics\texttrademark{}
% (далее COMSOL).

Целью работы является освоение методов математического моделирования
массопереноса в микрофлюидных системах.
В задачи работы входит создание моделей специфического связывания
% TODO
% целевого
аналита с:
\begin{itemize}
  \item модифицированной поверхностью микроканала, через который протекает раствор аналита:
    в двумерной и одномерной постановках,
    при наличии и отсутствии неспецифически связывающихся молекул в растворе,
    при наличии и отсутствии сайтов неспецифического связывания на поверхности;
  % \item модифицированной мембраной, через которую протекает раствор аналита:
  %   в предположении, что мембрана является областью пространства,
  %   где сайты связывания распределены равномерно, и
  %   в предположении, что мембрана является набором нитей разных форм,
  %   расположенных параллельно, на поверхности которых располагаются сайты связывания;
  \item модифицированной
    % TODO
    % целевыми комплементарными молекулами
    поверхностью микросфер, находящихся в растворе аналита
    в потоке и при отсутствии течения.

\end{itemize}


\Section{Обзор литературы}

\Subsection{Микрофлюидика}

Микрофлюидика --- наука о поведении малых объёмов жидкостей (от микро- до фемтолитров)
при малых потоках.
Малость объёмов и потоков можно определять значимостью эффектов масштаба:
ламинарные потоки, определяющая роль капиллярных явлений,
большое отношение площади поверхности к объёму.

Микрофлюидика может применяться\cite{bib:fabrication_and_application}:
\begin{itemize}
  \item для создания диагностических систем (\textbf{lab-on-a-chip})
        %(рис.~\ref{fig:diagnostic_example}),

  \item для создания и исследования культур клеток (\textbf{organ-on-a-chip}),

  \item для адресной доставки лекарственных веществ (drug delivery),

  \item для синтеза наноматериалов.

\end{itemize}
% \FloatBarrier{}

Для данной работы интерес представляют, в первую очередь,
диагностические системы.
Диагностическими здесь и далее называются системы, позволяющие обнаружить
(по возможности, определить/оценить количественно)
содержание конкретного вещества в некотором растворе
(зачастую --- в биологической жидкости, например, в крови, моче, слюне, лимфе и т. д.)
К возможным преимуществам использования
микрофлюидных диагностических систем относятся\cite{bib:evstrapov_mf_chip,bib:white_mf_as}:
% TODO bib:white_mf_as цитирован отвратительно --- исправить
\begin{itemize}
  \item работа с малыми объёмами исследуемой жидкости,

  \item использование малых количеств веществ-детекторов,

  \item минимизация потерь жидкости на стенках реактора в виду отсутствия
    необходимости переноса веществ между сосудами,

  \item увеличенная скорость диффузионного массопереноса,

  \item портативность,

  \item простота в использовании.

\end{itemize}
Последние два пункта являются ключевыми для создания диагностических устройств,
которые можно использовать вдали от оснащённых лабораторий и
квалифицированных специалистов
(\textbf{PoC} (point-of-care) devices).
Эти устройства могут обеспечивать быструю диагностику инфекционных заболеваний,
что способствует более эффективному наблюдению за распространением этих заболеваний
и более эффективной борьбе с ними\cite{bib:poc_infectious}.

Далее будем подразумевать, что ключом к детекции так или иначе
является специфическая адсорбция аналита,
что исключает из числа рассматриваемых диагностических методик, например, ПЦР.


% TODO примеры

% TODO элементы микрофлюидных систем, чипы



\Subsection{Микрофлюидные аналитические системы}

Примерами применения микрофлюидных аналитических систем являются
(капиллярный) электрофорез\cite{bib:capillary_protein_electrophoresis},
микрофлюидная хроматография\cite{bib:microfluidic_protein_chromatography},
цифровой ПЦР\cite{bib:droplet_microfluidics_review},
биосенсинг\cite{bib:biosensor_intro}.
К преимуществам использования таких систем относятся
малое количество исследуемого вещества,
малое количество веществ-детекторов,
высокая чувствительность,
а в случае с безмаркерным биосенсингом --- отсутствие влияния маркера
на исследуемую кинетику\cite{bib:marker_effect}.

Одной из первых микрофлюидных аналитических систем принято считать
представленный в 1990 году хроматограф\cite{bib:early_chromatograph}.
У этого хроматографа в чипе размером $5\times5\text{мм}$ была
расположена
%вытравлена
колонка с размерами
$6\text{мкм} \times 2\text{мкм} \times 15\text{см}$
со спиральной геометрией (рис.~\ref{fig:miniature_chromatograph_scheme}).
Теоретическая эффективность разделения составила 8000 и 25000 теоретических тарелок
при времени анализа 1 и 5 минут соответственно.

\begin{figure}
  \centering
  \includegraphics[width=.5\textwidth]{pic/miniature_chromatograph_scheme}

  \caption{%
    \label{fig:miniature_chromatograph_scheme}%
    Схема устройства миниатюрного жидкостного хроматографа, представленного в 1990 году;
    спиральная геометрия колонки (\textbf{посередине})
  }

\end{figure}

% TODO русский язык
% TODO ЛНЧ не существует
Зачастую микрофлюидные аналитические чипы
носят название лабораторий-на-чипе (ЛНЧ, LOC) ---
небольших устройств (чипов), содержащих
% представляют из себя
% лаборатории-на-чипе (ЛНЧ, LOC) ---
% небольшие устройства (чипы), содержащие
множественные механизмы, позволяющие работать с исследуемым образцом.
В идеале, все необходимые манипуляции с образцом должно быть возможно произвести
без помощи сторонних устройств.
ЛНЧ так же называют системами микрототального анализа (СМТА, $\mu$TAS).
Вопрос о соответствии реальных устройств названию <<лаборатория-на-чипе>>
оставим за пределами данной работы.

Существенным преимуществом микрофлюидных диагностических устройств
может являться возможность их применения в целях медицинской диагностики
вдали от лабораторий и специально обученного персонала --- там,
где находится пациент.
Такие устройства называют point-of-care (PoC) устройствами\cite{bib:PoC_trends}.

Примером служит устройство под названием
Sangia (Silver Amplified NeoGold ImmunoAssay)\cite{bib:poc_rsa},
схема работы которого изображена на рисунке~\ref{fig:diagnostic_example}.
Сборщик крови содержит помеченные золотом антитела
к простатическому специфическому антигену (ПСА);
кровь, в которой антитела связались с ПСА поступает в канал,
на стенках которого расположены такие же антитела ---
образуется <<сэндвич>> антитело-ПСА-антитело;
затем происходит реакция с серебрянным амплификатором,
приводящая к регистрируемому изменению оптической плотности.
Данное устройство способно обнаружить ПСА менее, чем за 15 минут,
что позволяет как обнаружить рак предстательной железы,
так и наблюдать за состоянием пацента после лечения.

% TODO russify
\begin{figure}[h]
  \centering
  % \includegraphics[width=.7\textwidth]{pic/diagnostic_device_example}
  \includegraphics[width=.9\textwidth]{pic/example_of_diagnostic_device}
  \caption{\label{fig:diagnostic_example}%
    Пример микрофлюидной диагностической системы --- PoC (point-of-care) устройства,
    способного обнаружить простатический специфический антиген (ПСА)
    менее, чем за 15 минут
  }

\end{figure}

% TODO uTAS, электрофорез и хроматография
% TODO добавить описания/примеры/возможности диагностических микрофлюидных систем



% \FloatBarrier{}
\Subsection{Биосенсоры}

Биосенсоры --- устройства для измерений, связанных с биохимическими реакциями,
обычно переводят количество аналита в зоне реакции
в регистрируемый сигнал.
К областям применения биосенсоров относятся
обнаружение болезней и токсинов,
мониторинг окружающей среды,
контроль качества воды и еды\cite{bib:biosensor_intro}
и другие (см. рис.~\ref{fig:biosensor_areas_of_application}\cite{bib:biomolecula_biosensors}).

\begin{figure}
  \centering
  \includegraphics[width=.6\textwidth]{pic/areas_of_application_of_biosensors}

  \caption{%
    \label{fig:biosensor_areas_of_application}%
    Области применения биосенсоров%,
    % источник: \href{https://biomolecula.ru/articles/lymeexpress-studenty-v-borbe-s-kleshchevymi-infektsiiami}
    %                {biomolecula.ru}
  }

\end{figure}

% TODO
Отдельного внимания заслуживают биосенсоры на поверхностных оптических волнах (ПВ).
К ним относятся как биосенсоры, основанные на
поверхностном плазмонном резонансе (ППР)\cite{bib:lrspr_review,bib:spr_simultaneous,bib:spr_biodetection},
так и более новые биосенсоры на поверхносных волнах в
одномерном фотонном кристалле (ОФК)\cite{bib:odpc_visualization}.
Принцип работы ППР-биосенсора изображён на рисунке~\ref{fig:spr_biosensor_principle}.
Световая волна, источник которой находится противоположной от исследуемого раствора
стороны металлической плёнки (ППР-биосенсор), % или ОФК (ОФК-биосенсор),
на некоторую глубину проникает в раствор;
связывание аналита с антителами на границе раздела фаз
локально изменяет показатель преломления раствора, что приводит к изменению
угла, под которым интенсивность отражённого излучения оказывается минимальной%
\cite{bib:klimov_nanoplasmonics}.

Преимущество ОФК-биосенсора перед ППР-биосенсором в том,
что глубина проникновения световой волны в раствор для него больше,
что позволяет исследовать реакции, происходящие
на большем удалении от поверхности.
ПВ-биосенсоры являются безмаркерными.


\begin{figure}
  \centering
  \includegraphics[width=.8\textwidth]{pic/spr_biosensor_principle}

  \caption{%
    \label{fig:spr_biosensor_principle}%
    Принцип работы биосенсора на поверхносных оптических волнах
    на примере биосенсора, основанного на поверхностном плазмонном резонансе;
    световая волна (\textbf{снизу}) на некоторую глубину проникает в раствор (\textbf{сверху}),
    связывание аналита с антителами на границе раздела фаз (\textbf{сверху справа})
    локально изменяет показатель преломления раствора, что приводит к изменению
    угла, под которым интенсивность отражённого света минимальна
  }

\end{figure}

Стоит также упомянуть КМ-биосенсоры (кварцевые микровесы).
В кристалле кварца (пьезоэлектрика) путём прикладывания переменного напряжения
($\sim$ 4--6 МГц)
возбуждаются вынужденные механические колебания и с высокой точностью
(за счёт высокой добротности $\sim$ $10^6$)
измеряется резонансная частота таких колебаний.
При связывании молекул исследуемого вещества с поверхностью кристалла
изменяется масса резонатора, а вместе с ней и измеряемая напрямую величина ---
резонансная частота (рис.~\ref{fig:qcm_rus})\cite{bib:qcm_odor}.

\begin{figure}
  \centering
  \includegraphics[width=.7\textwidth]{pic/qcm_rus}

  \caption{%
    \label{fig:qcm_rus}%
    Схема устройства КМ-биосенсора;
    связывание аналита с поверхностью сенсора приводит
    к изменению массы резонатора и, следовательно,
    к изменению резонансной частоты, измеряемой непосредственно
  }

\end{figure}


% Биосенсоры характеризуются чувствительностью, специфичностью
% К преимуществам от использования биосенсоров относят
% чувствительность, специфичность,
% а в случае безмаркерных биосенсоров --- отсутствие влияния маркера на исследуемую кинетику.


\FloatBarrier{}
\Subsection{Флуоресцентные микросферы}
Мультиплексный анализ --- вид иммуноферментного анализа,
позволяющий одновременно обнаруживать сразу несколько целевых маркеров.
Важная разновидность мультиплексного анализа основана на применении
флуоресцентных микросфер\cite{bib:fm_multiplex}.

Суть метода состоит в создании нескольких типов микросфер.
Разные типы содержат разные составы флуорофоров, задающие уникальные
спектральные адреса (коды) своим типам, и разные типы антител,
задающие соответствие типов микросфер специфически связываемым маркерам.
После инкубации в исследуемой жидкости со сферами,
часть из которых связалась со своими целевыми маркерами,
связываются молекулы сигнального флуорофора (напр., фикоэритрина).
Дальнейший анализ может быть проведён, например,
методами проточной цитометрии: каждый тип микросфер
флуоресценцией обнаруживает как тип маркера (по спектральному адресу микросферы),
так и наличие связавшегося маркера (по наличию излучения сигнального флуорофора)
(рис.~\ref{fig:fm_multiplex}).

\begin{figure}
  \centering
  \includegraphics[width=.7\textwidth]{pic/fm_multiplex}

  \caption{%
    \label{fig:fm_multiplex}%
    Мультиплексный анализ, основанный на применении флуоресцентных микросфер;
    флуоресценция показывает как спектральный адрес типа микросфер (код) и
    соответствующий тип маркера, так и само наличие связавшегося маркера
  }

\end{figure}


\Subsection{Математическое описание релевантных физическо-химических процессов}

\subsubsection*{Гидродинамика}
Рассмотрим течение жидкости с плотностью
$\rho$ $\left[\text{кг}/\text{м}^3\right]$
и скоростью течения $\vb v$ $\left[\text{м}/\text{с}\right]$.
Для изменения плотности во времени в элементарном объёме справедливо уравнение непрерывности
\begin{equation}
  \frac{\partial\rho}{\partial t} = - \div \rho \vb v,
\end{equation}
\begin{equation}
  \frac{\partial\rho}{\partial t} + \rho \div \vb v + \vb v \bcdot \grad \rho = 0.
\end{equation}
Как правило, воду (и водные растворы) считают несжимаемой жидкостью, полагая
$\rho = \const$, $\partial\rho / \partial t = 0$, $\grad \rho = 0$ $\Rightarrow$ $\div \vb v = 0$.

Когда на движущийся элементарный объём раствора действует сила
$\vb f$ $\left[\text{Н}/\text{м}^3\right]$,
согласно второму закону Ньютона полная производная имульса этого объёма по времени
(при неизменной плотности) равна
\begin{equation}
  \rho\frac{d \vb v}{d t} = \vb f.
\label{eq:pre_euler}
\end{equation}
Для перехода к частной производной скорости течения по времени в точке
$\left(x, y, z\right)$ достаточно раскрыть производную
\begin{equation}
  \frac{d}{dt} = \frac{\partial x}{\partial t}\frac{\partial}{\partial x} +
                 \frac{\partial y}{\partial t}\frac{\partial}{\partial y} +
                 \frac{\partial z}{\partial t}\frac{\partial}{\partial z} +
                 \frac{\partial}{\partial t}
               = \vb v \bcdot \bnabla + \frac{\partial}{\partial t},
\end{equation}
тогда~\eqref{eq:pre_euler} переходит в уравнение Эйлера
\begin{equation}
  \frac{\partial \vb v}{\partial t} + \left(\vb v \bcdot \bnabla\right) \vb v = \frac{1}{\rho} \vb f.
\label{eq:euler}
\end{equation}

Для ньютоновской несжимаемой жидкости при отсутствии дополнительных внешних сил
(гравитационная, электростатическая и т. п.)
$\vb f = - \grad P + \eta \Delta \vb v$,
$P$ --- гидростатическое давление $\left[\text{Па}\right]$,
$\eta$ --- динамическая вязкость жидкости $\left[\text{Па}\cdot\text{с}\right]$;
что приводит к уравнению Навье-Стокса\cite{bib:ll}
\begin{equation}
  \frac{\partial \vb v}{\partial t} + \left(\vb v \bcdot \bnabla\right) \vb v =
    \frac{1}{\rho}\left(-\grad P + \eta \Delta \vb v\right) =
    -\frac{\grad P}{\rho} + \nu \Delta \vb v,
\label{eq:navier-stokes}
\end{equation}
где $\nu$ --- кинематическая вязкость жидкости $\left[\text{м}^2/\text{с}\right]$.
В случае малых скоростей членом $\left(\vb v \bcdot \bnabla\right) \vb v$
в левой части можно пренебречь, как это сделано ниже, что даст уравнение Стокса
\begin{equation}
  \frac{\partial \vb v}{\partial t} = -\frac{\grad P}{\rho} + \nu \Delta \vb v.
\label{eq:stokes}
\end{equation}

\subsubsection*{Основная система уравнений}
Течение жидкости в микрофлюидных системах характеризуется малыми числами
Рейнольдса $\text{Re} = u L / \nu = \rho u L / \mu$,
$u$ --- скорость течения $\left[\text{м}/\text{с}\right]$,
$L$ --- характерный размер задачи $\left[\text{м}\right]$,
$\nu$ --- кинематическая вязкость жидкости $\left[\text{м}^2/\text{с}\right]$
$\rho$ --- плотность жидкости $\left[\text{кг}/\text{м}^3\right]$,
$\mu$ --- динамическая вязкость жидкости $\left[\text{Па}\cdot\text{с}\right]$.
Таким образом, поток в микроканалах можно считать ламинарным и использовать
уравнение Стокса~\eqref{eq:stokes}.

% TODO перекатать ЛЛ

Жидкость можно считать несжимаемой, что приводит к системе уравнений
\begin{equation}
\begin{cases}
  % \rho\frac{\displaystyle \partial \vb v}{\displaystyle \partial t} = -\nabla P + \eta\Delta\vb v + \vb f
  %   & \left(\text{уравнение Стокса}\right) \\
  % \nabla\cdot\vb v = 0 & \left(\text{несжимаемость}\right) \\
  % \frac{\displaystyle \partial c_i}{\displaystyle \partial t} = -\nabla\cdot\vb j_i + R_i
  %   & \left(\text{массоперенос}+\text{химия}\right) \\
  % \vb j_i = -D_i\nabla c_i + c_i \vb v
  %   & \left(\text{конвекция-диффузия}\right) \\
  \rho\frac{\displaystyle \partial \vb v}{\displaystyle \partial t} = -\grad P + \eta\Delta\vb v + \vb f
    & \left(\text{уравнение Стокса}\right) \\
  \div\vb v = 0 & \left(\text{несжимаемость}\right) \\
  \frac{\displaystyle \partial c_i}{\displaystyle \partial t} = -\div\vb j_i + R_i
    & \left(\text{массоперенос}+\text{химия}\right) \\
  \vb j_i = -D_i\grad c_i + c_i \vb v
    & \left(\text{конвекция-диффузия}\right) \\
\end{cases}
\label{eq:main}
\end{equation}
где $\rho$ --- плотность раствора $\left[\text{г}/\text{мл}\right]$,
$\vb v$ --- скорость ламинарного течения $\left[\text{мм}/\text{с}\right]$,
$P$ --- гидростатическое давление $\left[\text{Па}\right]$,
$\eta$ --- динамическая вязкость раствора $\left[\text{Па}\cdot\text{с}\right]$,
$\vb f$ --- внешняя сила, действующая на элемент объёма раствора
$\left[\text{дина}/\text{мл}\right]$
(например, сила тяжетси),
$c_i$ --- концентрация $i$-ого вещества в растворе $\left[\text{мМ}\right]$,
$\vb j_i$ --- поток $i$-ого вещества в растворе
$\left[\text{мМ}/\left(\text{мм}^2\cdot\text{с}\right)\right]$,
$R_i$ --- изменение концентрации $i$-ого вещества в растворе, связанное с
химическими реакциями $\left[\text{мМ}/\text{с}\right]$.

По всей видимости, (существенных для микрофлюидики) границ применимости
у системы~\eqref{eq:main} две:
\begin{enumerate}
  \item концентрации растворённых веществ должны быть достаточно большими,
    чтобы было допустимым не переходить к статистическому описанию
    движения их молекул, а остаться в рамках непрерывного приближения,
    % TODO а что с очент большими концентрациями???

  \item характерный размер $\lambda$ (ширина) микрофлюидного канала должнен
    значительно превосходить длины
    свободного пробега молекул, чтобы течение жидкости было вязкостным,
    а не переходным или молекулярным (Кнудсеновским);
    на практике, для водоподобных сред, это ограничение
    $\lambda\gtrsim300\text{нм}$.

\end{enumerate}

\subsubsection*{Граничные условия}
Далее приняты следующие обозначения: $\Omega$ --- расчётная область,
$\mathcal{W} \subset \partial\Omega$ --- стенка
(на ней обнуляется скорость течения),
$\mathcal{S}_i \subset \mathcal{W}$ --- часть стенки,
содержащая сайты связывания $i$-ого вещества,
$\mathcal{I} \subset \partial\Omega$ --- входное отверстие (inlet),
$\mathcal{O} \subset \partial\Omega$ --- выходное отверстие (outlet).

Граничные условия для системы~\eqref{eq:main} на стенке $\mathcal{W}$:
\begin{equation}
\begin{cases}
  \left. \vb v \right|_{\mathcal{W}} = \vb 0 \\
  \left. \vb n \bcdot \vb j_i \right|_{\mathcal{W}} = r_i\chi_{\mathcal{S}_i}
\end{cases}
\label{eq:main:bc_walls}
\end{equation}
где $r_i$ --- скорость адсорбции $i$-ого вещества,
$\chi_{\mathcal{S}_i}$ --- характеристическая функция части стенок,
на которой адсорбция вообще происходит,
$\vb n$ --- поле единичных внешних нормалей к $\partial \Omega$.
Такое описание несколько избыточно: действительно, можно сделать замену
$r_i = r_i\chi_{\mathcal{S}_i}$.

Граничные условия на входном/выходном отверстиях $\mathcal{I}$/$\mathcal{O}$:
\begin{equation}
\begin{cases}
  \left. \vb v \right|_{\mathcal{I}/\mathcal{O}}\left(t, x, y, z\right) \\
  \left. P \right|_{\mathcal{I}/\mathcal{O}}\left(t, x, y, z\right) \\
  \left. c_i \right|_{\mathcal{I}}\left(t, x, y, z\right) \\
  \left. \vb n \bcdot \grad c_i \right|_{\mathcal{O}} = 0 \\
\end{cases}
\label{eq:main:bc_io}
\end{equation}
здесь отсутствие правой части равенства подразумевает, что зафиксирована
(известна) <<стоящая в левой части равенства>> функция.


\subsubsection*{Изотермы адсорбции}
Под изотермой адсорбции будем понимать всякое уравнение
$F\left(c_{i=\overline{1,n}}, \gamma_{i=\overline{1,n}}\right) = 0$, где
$c_i$ --- равновесная концентрация $i$-ого вещества в растворе,
$\gamma_i$ --- равновесная (поверхностная) концентрация $i$-ого вещества
на поверхности, или параметрические семейства таких уравнений,
параметрами в которых будут такие величины как
поверхностная концентрация сайтов связывания и
равновесные химические константы (например, константа диссоциации).
Дополнительно потребуем, чтобы такие уравнения задавали функции
$\gamma_i\left(c_{i=\overline{1,n}}, \gamma_{j\neq i}\right)$.

Изотермы адсорбции здесь представляют интерес с точки зрения получения
выражений для скоростей реакций $r_i$ в
граничных условиях~\eqref{eq:main:bc_walls}.
Они также могут использоваться для приближений.

Простейшей (не считая изотерму адсорбции Генри) изотермой адсорбции является
изотерма \textbf{Ленгмюра} для одного вещества
с единственным видом сайтов связывания
с поверхностной концентрацией $\Gamma$
\begin{equation}
  \gamma = \frac{\Gamma c}{K_d + c}
\label{eq:Langmuir_single_eq}
\end{equation}%
\cite{bib:adsorption_reaction_principles},
где $K_d$ --- константа диссоциации вещества и сайтов связывания на поверхности.
Изотерме Ленгмюра соответствует кинетика, описываемая законом действующих масс
% TODO закон действующих масс-поверхностей?
\[ A_{\left(\text{bulk}\right)} + S_{\left(\text{surf}\right)} \xrightleftharpoons[k_r]{k_f} AS_{\left(\text{surf}\right)} \]
\begin{equation}
  r = k_f c \left(\Gamma - \gamma\right) - k_r \gamma,\qquad K_d = \frac{k_r}{k_f}.
\label{eq:Langmuir_single_kin}
\end{equation}

Усложнением будет многокомпонентная изотерма Ленгмюра с единственным видом
сайтов связывания
\begin{equation}
  \gamma_i = \frac{\Gamma K_{a,i} c_i}{1 + \sum\limits_{j=1}^{n} K_{a,j} c_j},
\label{eq:Langmuir_multi_eq}
\end{equation}
где $K_{a,i} = 1/K_{d,i}$ --- константа аффинности $i$-ого вещества и
сайтов связывания. Этой изотерме соответствует кинетика,
схожая с~\eqref{eq:Langmuir_single_kin}
\begin{equation}
  r_i = k_{f,i} c_i \left(\Gamma - \sum_{j=1}^{n} \gamma_j\right) - k_{r,i}\gamma_i =
        k_{f,i} c_i \Gamma_{\text{free}} - k_{r,i}\gamma_i.
\label{eq:Langmuir_multi_kin}
\end{equation}

При наличии $m$ видов сайтов связывания изотерма Ленгмюра усложнится до
\begin{equation}
  \gamma_i = \sum_{k=1}^{m}\frac{\Gamma^k K_{a,i}^k c_i}{1 + \sum\limits_{j=1}^{n} K_{a,j}^k c_j},
\label{eq:Langmuir_multi_many_eq}
\end{equation}
а кинетика --- до
\begin{equation}
  r_i = \sum_{k=1}^{m} \left[
            k_{f,i}^k c_i \left(\Gamma^k - \sum_{j=1}^{n} \gamma_j^k\right) - k_{r,i}^k\gamma_i^k
          \right] =
        \sum_{k=1}^{m} \left[
            k_{f,i}^k c_i \Gamma_{\text{free}}^k - k_{r,i}^k\gamma_i^k
          \right].
\label{eq:Langmuir_multi_many_kin}
\end{equation}

% TODO когда Ленгмюр не годится

Другой пример изотермы адсорбции ---
\textbf{БЭТ}-изотерма (Брунауэр-Эммет-Теллер).
Эта изотерма, в отличие от Ленгмюровской, описывает полислойную адсорбцию.
При наличии единственного вида сайтов связывания и единственного аналита
скорость образования $i$-ого адсорбционного слоя считается равной
\begin{equation}
  r_i = k_{f,i} c \gamma_{i-1} - k_{r,i} \gamma_{i},
\label{eq:BET_ith_layer_deriv}
\end{equation}
где $\gamma_{i \neq 0}$ --- поверхностная концентрация аналита в
$i$-ом адсорбционном слое, а
$\gamma_0$ --- концентрация свободных сайтов связывания.
Скорость адсорбции в таком случае равна $r = \sum_{i=1}^{\infty} r_i$.

Если предположить, что
% TODO записать по-русски???
$\forall i \in \mathbb{N} \quad i > 1 \Rightarrow
%\left[2,\infty\right)
k_{f,i}/k_{r,i} = K_a = \text{const}$,
то в равновесии получится уравнение изотермы БЕТ
\begin{equation}
  \gamma = \sum_{i=1}^{\infty}\gamma_i =
    \Gamma\frac{ck_{f,1}/k_{r,1}}
               {\left(1 - cK_a\right)
                \left[1 - c\left(K_a - k_{f,1}/k_{r,1}\right)\right]}.
\label{eq:BET_isotherm}
\end{equation}

Изотерма \textbf{Фрейндлиха} описывает ситуацию,
когда энергия адсорбции распределена по сайтам связвания неравномерно.
Изотерма Фрейндлиха --- эмпирическое соотношение,
что затрудняет его физическую интерпретацию.
% TODO Зельдович

Сама изотерма:
\begin{equation}
  \gamma = K^*c^{1/n},
\label{eq:pure_Freundlich_isoterm}
\end{equation}
где $K^*$, $n>1$ --- константы для фиксированных веществ при фиксированной температуре.
Учёт насыщения адсорбирующей поверхности приводит к
уравнению изотермы Фрейндлиха-Ленгмюра
\begin{equation}
  \gamma = \Gamma\frac{c^{1/n}}{K + c^{1/n}},
\label{eq:FL_isoterm}
\end{equation}
где $K$ --- константа, равная концентрации $c_{1/2}$,
при которой занята половина сайтов связывания, возведённой в степень $1/n$.

По всей видимости,
если потребуется, имеет смысл считать, что
кинетика, соответствующая изотерме~\eqref{eq:FL_isoterm}
имеет вид\cite{bib:freundlich_kinetics}
\begin{equation}
  r = k_f c^{1/n} (\Gamma - \gamma) - k_r \gamma.
\label{eq:FL_kinetics}
\end{equation}



\subsubsection*{Приближения}
Решение системы~\eqref{eq:main} в сложной геометрической области
не всегда целесообразно,
в связи с чем для упрощения задачи моделирования
может быть использован ряд приближений.
В частности, решение обратных задач для ОДУ
(обыкновенных дифференциальных уравнений) значительно проще, чем для УрЧП
(уравнений в частных производных),
так что, например, для определения кинетических параметров исследуемой системы
по данным, полученным с помощью биосенсора, предпочтительнее
может оказаться использование
приближения, описывающего исследуемую систему как ОДУ, а не как УрЧП%
\cite{bib:FULLTEXT_inverse_example}.

Простейшее приближение состоит в пренебрежении диффузионными и конвекционными
процессами. Реактор идеально перемешан и описывается системой ОДУ
\begin{equation}
\begin{cases}
  \dot{\gamma}_{i,j} = r_{i,j}\\
  \dot{c}_i = \frac{S}{V}\sum\limits_{j=1}^{m}-r_{i,j}
\label{eq:perfectly_mixed_ode}
\end{cases}
\end{equation}
где $m$
$\gamma_{i,j}$ --- поверхностная концнетрация $i$-ого вещества,
адсорбировавшегося на $j$-ом виде сайтов связывания,
$c_i$ --- объёмная концентрация $i$-ого вещества,
$r_{i,j}$ --- скорость адсорбции $i$-ого вещества на $j$-ом виде сайтов связывания,
$S$ и $V$ --- площадь поверхности и объём реактора соответственно.
Всего в реакторе $n$ веществ, а на его поверхности $m$ видов сайтов связывания.

Более сложным приближением является \textbf{модель двух компартментов}
(two-compartment model (\textbf{TCM}))\cite{bib:TCM}.
Пространство реактора разделяется на две области (два компартмента):
внешний --- с постоянной концентрацией аналита $c_0$, и
внутренний --- с концентрацией аналита $c$.
Скорость обмена аналитом между компартментами полагается равной
\begin{equation}
  v_\text{ex} = k_m \left(c_0 - c\right)
\label{eq:TCM:exchange}
\end{equation}
где $k_m = 1.282\sqrt[3]{D^2 v_\text{fl} / \left(L_s h_c\right)}\;
  \left[\text{м}/\text{с}\right]$, 
$v_\text{fl}$ --- скорость течения раствора,
$L_s$ --- характерный размер адсорбирующей поверхности,
$h_c$ --- высота внутреннего компартмента.
Скорость адсорбции полагается равной
\begin{equation}
  \dot{\gamma} = k_f c \left(\Gamma - \gamma\right) - k_r\gamma,
\label{eq:TCM:adsorption}
\end{equation}
а скорость изменения концентрации аналита во внутреннем компартменте ---
\begin{equation}
  \dot{c} = \frac{v_\text{ex} - \dot{\gamma}}{h_c} =
            k_m^* \left(c_0 - c\right) - k_f c \left(\Gamma^* - \gamma^*\right) + k_r \gamma^*,
\label{eq:TCM:inner}
\end{equation}
где $k_m^* = k_m/h_c$, $\Gamma^* = \Gamma/h_c$, $\gamma^* = \gamma/h_c$.
% TODO а как определять h_c???

\subsubsection*{Вязкостный и диффузионный слои}
В силу граничных условий, скорость течения жидкости около стенки равна нулю.
В связи с этим вводят понятие вязкостного слоя, одна из границ которого
совпадает с границей расчётной области, прилегающей к стенке,
а около другой границы скорость течения практически не меняется в пространстве
и совпадает со скоростью течения вне вязкостного слоя.

Аналогично при наличии поглощения вещества на поверхности вводится понятие
диффузионного слоя, не имеющее прямого отношения к аналогичному понятию из
электрохимии.
Аналогом величины скорости выступает концентрация реагента в объёме жидкости.

Вышесказанное проиллюстрировано на рисунке~\ref{fig:visc_diff_layers}.
% взятом из~\cite{bib:phys_chem_hydro_layers}.
% Там же есть обоснование формул для оценки толщин этих слоёв:
Существуют формулы, позволяющие оценивать толщины вязкостного и диффузионного
слоёв $\delta_U$ и $\delta_D$ соответственно\cite{bib:ll,bib:phys_chem_hydro_layers}
\begin{equation}
  \delta_U \sim \sqrt{\frac{L\nu}{U}},
\label{eq:viscous_layer}
\end{equation}
\begin{equation}
  \delta_D \sim \sqrt[3]{\frac{D}{\nu}} \delta_U.
\label{eq:diffusion_layer}
\end{equation}
Здесь $U$ --- скорость течения вне вязкостного слоя,
$L$ --- характерный размер задачи (характерная длина вдоль потока),
$\nu$ --- кинематическая вязкость текущей жидкости,
$D$ --- коэффициент диффузии растворённого реагирующего с поверхностью вещества.
Отметим, что коэффициент $\sqrt[3]{D/\nu}$, связывающий $\delta_D$ с $\delta_U$
обычно имеет порядок $0.1$.


\begin{figure}
  \centering
  \includegraphics[width=.7\textwidth]{pic/visc_diff_layers}
  \caption{\label{fig:visc_diff_layers}%
    Вязкостный и диффузионный слои;
    $\delta_U$ и $\delta_D$ --- их толщины соответственно
  }

\end{figure}



\Subsection{Моделирование в микрофлюидике}
Математическое моделирование микрофлюидных систем может быть использовано для:
\begin{itemize}
  \item предсказания свойств чипа до его производства,
    что удешевляет разработку конечного функционального изделия,

  \item интерпретации данных, получаемых с помощью чипа,

  \item определения свойств исследуемой системы
    (путём решения обратных задач).

\end{itemize}
% TODO больше о моделировании

% TODO Притянуть стоимость микрофабрикации

\subsubsection*{Решение системы~\eqref{eq:main}}
Как правило, говорить об аналитическом решении системы~\eqref{eq:main}
не приходится и система решается численно,
например, методами конечных элементов.

Методы конечных элементов --- семейство методов численного решения
уравнений математической физики, состоящие в разбиении расчётной области
на конечное число подобластей --- конечных элементов,
на которых выбираются базисные функции,
равные нулю всюду кроме своих элементов
(или конечного их числа, смотря что иметь в виду под элементом)\cite{bib:comsol_fem},
а решение ищется в виде линейной комбинации этих функций.
Это проиллюстрировано на рисунке~\ref{fig:basis_functions_FEM_illustration1d}.

\begin{figure}
  \centering
  \includegraphics[width=.5\textwidth]{pic/basis_functions_FEM_illustration1d}
  \caption{\label{fig:basis_functions_FEM_illustration1d}%
    Аппроксимация функции $u\left(x\right)$ линейной комбинацией
    финитных базисных функций $\psi_{i=\overline{0,9}}\left(x\right)$
    % (\href{https://cdn.comsol.com/cyclopedia/finite-element-method/plot-using-discretization.png}
    %       {www.comsol.com/multiphysics/finite-element-method})
  }

\end{figure}

На примере уравнения Пуассона
\begin{equation} \Delta u = f\left(\vb x\right) \label{eq:poisson_strong}\end{equation}
выберем гильбертово пространство $H$ и будем искать
$u: \Omega \rightarrow \mathbb{R} \in H: \forall \psi \in H$
\begin{equation}
  \int\limits_\Omega \psi \Delta u d\vb x =
  \int\limits_{\partial\Omega} \psi \grad u \bcdot \vb{dS} -
    \int\limits_\Omega \grad\psi \bcdot \grad u d\vb x =
  \int\limits_\Omega \psi f d\vb x.
\label{eq:poisson_weak}
\end{equation}
Здесь~\eqref{eq:poisson_weak} --- слабая (вариационная) формулировка,
$\psi$ --- пробная функция.

При применении методов конечных элементов в качестве $H$ выбирается линейная
оболочка набора финитных бызисных функций $\psi_i$, о которых речь шла выше.
Решение ищется в виде их линейной комбинации $u = \sum_i u_i\psi_i$
Уравнение~\eqref{eq:poisson_weak} заменяется на систему (по индексу $j$)
\begin{equation}
  \sum_i u_i \left(
    \int\limits_{\partial\Omega} \psi_j \grad \psi_i \bcdot \vb{dS} -
    \int\limits_\Omega \grad\psi_j \bcdot \grad \psi_i d\vb x
  \right) =
  \int\limits_\Omega \psi_j f d\vb x,
\label{eq:poisson_FEM}
\end{equation}
что можно записать в виде
\begin{equation}
  \hat{A}\vb u = \vb f,
\label{eq:poisson_FEM_obscure}
\end{equation}
где $\vb u$ --- столбец с элементами $u_i$,
$\vb f$ --- столбец с элементами $f_i = \int_\Omega\psi_i f d\vb x$,
$\hat{A}$ --- матрица с элементами $A_{ij} =
  \int_{\partial\Omega} \psi_i \grad \psi_j \bcdot \vb{dS} -
    \int_\Omega \grad\psi_i \bcdot \grad \psi_j d\vb x$.
Таким образом, численное решение уравнения Пуассона методами конечных элементов
запишется как
\begin{equation}
  u\left(\vb x\right) =
    \sum_i\left[\hat{A}^{-1}\vb f\right]{}_i\psi_i\left(\vb x\right).
\end{equation}

В случае с уравнением теплопроводности
\begin{equation}
  \frac{\partial u}{\partial t} - \Delta u =
    f\left(\vb x, t, u\left(\vb x, t\right)\right)
\label{eq:heat_strong}
\end{equation}
имеет смысл (для уменьшения вычислительных затрат)
искать приближённое решение в виде
$u = \sum_i u_i\left(t\right)\psi_i\left(\vb x\right)$
(вместо $\sum_i u_i\psi_i\left(\vb x, t\right)$).
В таком случае уравнение~\eqref{eq:heat_strong} будет приближаться
(аналогично~\eqref{eq:poisson_FEM})
\begin{equation}
  \sum_i \frac{\partial u_i}{\partial t} \int\limits_\Omega \psi_i\psi_j d\vb x +
    \sum_i u_i \left(
      -\int\limits_{\partial\Omega} \psi_j \grad \psi_i \bcdot \vb{dS} +
      \int\limits_\Omega \grad\psi_j \bcdot \grad \psi_i d\vb x
    \right) =
  \int\limits_\Omega \psi_j f_t d\vb x,
\label{eq:heat_FEM}
\end{equation}
где $f_t\left(\vb x\right) =
  f\left(\vb x, t, \sum_i u_i\psi_i\left(\vb x\right)\right)$.
Выражение $\partial u_i/\partial t$ заменится конечной разностью, например
\begin{equation}
  \frac{\partial u_i}{\partial t} \approx
    \frac{u_i\left(t + \Delta t\right) - u_i\left(t\right)}{\Delta t},
\label{eq:heat_time_forward_finite_difference}
\end{equation}
в таком случае подстановка $u_i = u_i\left(t\right)$ в~\eqref{eq:heat_FEM}
позволит явно выразить $u_i\left(t + \Delta t\right)$ через $u_i\left(t\right)$:
\begin{equation}
  \vb u\left(t + \Delta t\right) =
    \vb u\left(t\right) +
      \Delta t\hat{\Psi}^{-1} \left(\hat{A}\vb u\left(t\right) +
                                    \vb f_t\left(\vb u\left(t\right)\right)\right),
\label{eq:heat_FEM_explicit}
\end{equation}
где $\hat{\Psi}$ --- матрица с элементами $\Psi_{ij} = \int_\Omega\psi_i\psi_j d\vb x$,
$\vb u(t)$ --- столбец с элементами $u_i\left(t\right)$,
$\vb f_t\left(\vb u\left(t\right)\right)$ ---
  столбец с элементами $f_{t,i} = \int_\Omega\psi_i f_t d\vb x$,
$\hat{A}$ определена так же, как и для уравнения Пуассона выше.

Если пользоваться~\eqref{eq:heat_time_forward_finite_difference},
и принять $u_i = u_i(t + \Delta t)$ в~\eqref{eq:heat_FEM},
то полученное уравнение будет задавать
$u_i\left(t + \Delta t\right)$ как функцию $u_i\left(t\right)$ неявно.
Неявная постановка вычислительно более затратна, но
оказывается необходимой при решении так называемых \emph{жёстких} систем\cite{bib:hairer_stiff},
которые часто встречаются при наличии химических реакций
(когда уравнение теплопроводности является, по сути, уравнением диффузии).

Методы конечных элементов реализованы, например, в программном обеспечении
COMSOL Multiphysics\texttrademark\ (далее --- COMSOL),
которое и используется в этой работе.

\subsubsection*{Решение обратных задач}
Как было сказано выше, для определения характеристик исследуемых систем
могут решаться обратные задачи.
Если решение прямой задачи состоит в предсказании поведения системы,
про которую всё известно, то решение обратной задачи состоит в оценке
параметров системы по её поведению\cite{bib:inverse_pde}.

% Обратные задачи математически формулируются как задачи оптимизации ---
Общий способ численного решения обратных задач состоит в
минимизации некоторого функционала ошибки $Q:\mathbb{P}\rightarrow\mathbb{R}_+$,
где $\mathbb{P}$ --- пространство параметров,
$\mathbb{R}_+ = \left\{x\in\mathbb{R}|x\geqslant0\right\}$.
Если $\vb x_{t}$ --- состояние системы в момент времени $t$, а
$\vb f\left(t,\vb p\right)$ --- предсказание состояния системы с параметрами
$\vb p \in \mathbb{P}$ в момент времени $t$, то
типичным функционалом ошибки будет
\begin{equation}
Q\left(\vb p\right) =
  \frac{\sum\limits_{t\in T}\left(\vb f\left(t,\vb p\right) - \vb x_t\right){}^2}
       {\left|T\right|},
\end{equation}
где $T$ --- конечное множество времён.
Данный функционал гладкий, что позволяет для его оптимизации использовать
градиентные методы; если аналитическое выражение для градиента не известно,
он может быть в каждой точке оценён численно.

Например, при адсорбции из идеально перемешанного реактора
согласно кинетике действующих масс, соответствующей изотерме Ленгмюра,
поверхностная концентрация адсорбированного аналита
в момент времени $t$ будет равна
\begin{equation}
  \gamma\left(t\right) = f\left[c\right]\left(\Gamma_0, k_f, k_r, t\right),
\end{equation}
где $c:T\rightarrow\mathbb{R}_+$ --- концентрация аналита в растворе в зависимости от времени,
$\Gamma_0$ --- поверхностная концентрация сайтов связывания,
$k_f$ и $k_r$ --- кинетические константы, характеризующие адсорбцию и десорбцию,
$f$ --- решение уравнения
$\dot{\gamma} = k_f \left(\Gamma_0-\gamma\right) c - k_r \gamma$,
которое может быть получено численно.
Пусть теперь в каждый момент времени известна концентрация $c\left(t\right)$
(например, через реакционную камеру протекают растворы с известными концентрациями)
и имеется линейный по поверхностной концентрации адсорбировавшегося аналита
сигнал с прибора $s\left(t\right) = \alpha\gamma\left(t\right)$.
Тогда для сигнала будет справедливо
\begin{equation}
\begin{cases}
  \dot{s} = \alpha\dot{\gamma} = k_f \left(s_{\max} - s\right) - k_r s \\
  s = s_t = s_0 + \int\limits_{t_0}^{t} \dot{s} dt = g(t, s_{\max}, k_f, k_r) \\
\end{cases}
\end{equation}
и по набору значений $s_{t \in T}$ можно будет оценить параметры
$s_{\max}$, $k_f$ и $k_r$\cite{bib:FULLTEXT_inverse_example}
(рис.~\ref{fig:FULLTEXT_inverse_example}).

\begin{figure}
  \centering
  \includegraphics[width=.8\textwidth]{pic/FULLTEXT_inverse_example}
  \caption{\label{fig:FULLTEXT_inverse_example}%
    Решение обратной задачи с помощью ClampXP:
    в левом окне задаётся концентрация аналита в растворе от времени,
    в правом окне внизу слева полученные оценки кинетических констант
  }

\end{figure}


% TODO каплегенераторы

% TODO переимновать
% \Section{Основное содержание}
\Section{Модели}

\Subsection{Описание задач}
\subsubsection*{Адсорбция на поверхность бесконечно высокого канала с одним видом сайтов связывания}

Имеется раствор вещества A (аналит),
в котором так же может присутствовать вещество B (примесь).
На плоской поверхности канала,
вдоль которой течёт этот раствор,
имеются сайты связывания.
Это проиллюстрировано на рисунке~\ref{fig:flat_wide_illustration}
(см. также рис.~\ref{fig:visc_diff_layers}).

\begin{figure}[h]
  \centering
  \includegraphics[width=.4\textwidth]{pic/flat_illustration}
  \caption{\label{fig:flat_wide_illustration}%
    Иллюстрация к задаче об адсорбции на поверхность;
    цветные кривые изображают зависимость
    скорости течения и концентрации аналита
    от расстояния до поверхности;
    оранжевые круги --- сайты связывания, связавшиеся с аналитом,
    салатовые --- свободные сайты связывания
  }

\end{figure}

Кинетика связывания соотвестсвует изотерме Ленгмюра
(см.~\eqref{eq:Langmuir_multi_many_kin}).

\subsubsection*{Адсорбция на поверхность бесконечно высокого канала с двумя видами сайтов связывания}

Задача во многом аналогична предыдущей. В этот раз всегда имеется лишь одно вещество,
способное связываться с модифицированной поверхностью канала.
Однако видов сайтов связывания в этом случае два:
сайты специфического связывания (A) и сайты неспецифического связывания (B).

% TODO картин очка

\subsubsection*{Адсорбция на поверхность канала конечной высоты с двумя видами сайтов связывания}

В предыдущих задачах предполагалось, что объём раствора можно считать бесконечным ---
ограниченным лишь со стороны одной поверхности канала.
В данном случае канал имеет конечную высоту, что ограничевает рассматриваемый объём.
Основное отличие в том, что профиль скоростей
(с точностью до умножения на среднюю/максимальную скорость течения)
задаётся геометрией канала, в то время как
в предыдущих задачах толщина диффузионного слоя связана
с формулой~\eqref{eq:viscous_layer} и зависит от скорости течения.

% TODO картин очка

С точки зрения химии задача полностью аналогична предыдущей.


\subsubsection*{Адсорбция на микросферах}

В растворе с аналитом находятся в виде взвеси микросферы,
на поверхности которых находятся сайты связывания.
Сферы могут покоиться относительно раствора,
а могут равномерно поступательно двигаться
(например, сфера зафиксирована ловушкой, обтекается раствором).
Броуновским движением пренебрегается.

% TODO картин очка


\Subsection{Адсорбция на поверхность бесконечно высокого канала с одним видом сайтов связывания}

\subsubsection*{Одномерная постановка без примеси}
В качестве расчётной области берётся отрезок с длиной $\delta_D$
диффузионного слоя (см.~\eqref{eq:diffusion_layer}).
На одном конце отрезка происходит химическая реакция
связывания веществ из раствора с сайтами связывания на поверхности,
на другом --- фиксируется концентрация $c_0$.

Изначально концентрация веществ в растворе однородна (всюду равна $c_0$) и
все сайты связывания свободны (на поверхности нет связвашегося аналита).

Толщину вязкостного слоя $\delta_U$ можно представить в виде
\begin{equation}
  \label{eq:viscous_layer_alpha}
  \delta_U = \alpha\sqrt{\frac{L\nu}{U}},
\end{equation}
где $\alpha$ --- безразмерный параметр порядка единицы.
В этой части в формуле~\eqref{eq:diffusion_layer} для расчёта $\delta_D$
знак $\sim$ заменён на знак равенства, а
$\alpha$ в~\eqref{eq:viscous_layer_alpha} принимается равным $\sqrt{2}$,
последнее будет обосновано ниже.

Параметры задачи:
скорость течения раствора равна $U = 0.1\text{мм}/\text{с}$,
характерная длина $L = 2\text{мм}$,
вязкость раствора равна вязкости воды (\SI{20}{\celsius}).
В качестве аналита принимается стрептавидин,
его коэффициент диффузии оценён в $D = 8\cdot10^{-5}\text{мм}^2/\text{с}$,
а молекулярная масса --- в $m_{\text{str}} = 60\text{кДа}$.
Кинетические константы были приняты равными%
\label{streptavidin_kinetics}
$k_f = 10^7\left(\text{М}\cdot\text{с}\right)^{-1}$
и $k_r = 10^{-7}\text{с}^{-1}$ в соответствии с оценкой из~%
\cite{bib:biotin_streptavidin_kinetics}
кинетики связывания комплекса биотин-стрептавидин.
Поверхностная концентрация сайтов связывания принята равной%
\label{Gamma0.7}
$\Gamma = 3.85\cdot10^{-8}\text{моль}/\text{м}^2 = 0.023\text{нм}^{-2} = 1/\left(43.5\text{нм}^2\right)$
(связывание со всеми сайтами будет соответствовать примерно
70\% заполнения всей поверхности молекулами стрептавидина).

На рисунке~\ref{fig:flat_wide_onecomp_onedim_probe_distrib}
представлены результаты симуляций со значениями концентрации
$c_0 = 0.1 \text{мг}/\text{мл}$ и $c_0 = 0.01 \text{мг}/\text{мл}$.
Во втором случае диффузионный слой успевает истощиться до насыщения поверхности,
после 100--200 секунд с нулевого момента времени адсорбции лимитируется
максимальной возможной скоростью диффузии через диффузионный слой
$j_{\text{max}} = D c_0 / \delta_D$.

\begin{figure}
  \centering
  \includegraphics[width=.5\textwidth]{pic/flat_wide_onecomp_onedim_ref_probe}%
  \includegraphics[width=.5\textwidth]{pic/flat_wide_onecomp_onedim_dil10_probe}

  \includegraphics[width=.5\textwidth]{pic/flat_wide_onecomp_onedim_ref_distrib}%
  \includegraphics[width=.5\textwidth]{pic/flat_wide_onecomp_onedim_dil10_distrib}

  \caption{\label{fig:flat_wide_onecomp_onedim_probe_distrib}%
    Результаты расчётов в одномерной задаче об адсорбции
    на стенку бесконечно высокого канала;
    \textbf{сверху:} зависимость концентрации связвашегося стрептавидина от времени,
    \textbf{снизу:} пространственное распределение стрептавидина по объёму раствора;
    \textbf{слева:} $c_0 = 0.1 \text{мг}/\text{мл}$,
    \textbf{справа:} $c_0 = 0.01 \text{мг}/\text{мл}$
  }

\end{figure}

\subsubsection*{Одномерная постановка с примесью}
Относительно стрептавидина, задача идентична предыдущей
с $c_0 = 0.01 \text{мг}/\text{мл}$.
Добавляется примесь с начальной и граничной концентрациями
$c_1 = 100 c_0$.
Константа диссоциации примеси и сайтов связывания равна
$K_{B,d} = 10^{-3} \text{М}$,
для кинетической константы $k_{B,f}$ рассматриваются значения
$k_{B,f} = 10^3 \left(\text{М} \cdot \text{с}\right)^{-1}$ и
$k_{B,f} = 10^4 \left(\text{М} \cdot \text{с}\right)^{-1}$.
Полученные зависимости поверхностных концентраций
стрептавидина, примеси и свободных сайтов связывания от времени
приведены на рисунке~\ref{fig:flat_wide_twobulk_paramsweep}.

\begin{figure}
  \centering
  \includegraphics[width=.6\textwidth]{pic/flat_wide_two_bulk_paramsweep_3}
  \includegraphics[width=.6\textwidth]{pic/flat_wide_two_bulk_paramsweep_4}

  \caption{%
    \label{fig:flat_wide_twobulk_paramsweep}%
    Одномерная постановка задачи об адсорбции на стенку бесконечно высокого канала
    при наличии примеси;
    зависимость поверхностных концентраций стрептавидина, примеси и
    свободных сайтов связывания от времени
  }
\end{figure}



\subsubsection*{Двумерная постановка (без примеси)}

Расчётная область представляет из себя прямоугольник со сторонами
$a = 3 L$ и $b = 2 \delta_U$.
Значение $\delta_U$ расчитывается по формуле~\eqref{eq:viscous_layer}
с заменой знака $\sim$ на $=$.
Значения всех параметров те же, что и в одномерной постановке;
$c_0 = 0.01 \text{мг}/\text{мл}$.

На поверхности канала (сторона $a$, нижняя граница расчётной области)
сайты связывания расположены на отрезке
длиной $L = 2 \text{мм}$, отстоящем от обоих концов рассматриваемой части поверхности канала
на расстоянии $L$ (см. рис.~\ref{fig:flat_wide_plate_illustration} сверху слева).
Вдоль левой границы расчётной области скорость потока равна
$U = 0.1 \text{мм}/\text{с}$ и направлена вдоль поверхности канала.
На верхней границе расчётной области концентрация зафиксирована и равна
$c_0 = 0.01 \text{мг}/\text{мл}$.

На рисунке~\ref{fig:flat_wide_plate_illustration} изображены
распределение концентрации стрептавидина в части объёма раствора
в пределах $2 \delta_D$ от стенки канала
(в прямоугольнике со сторонами $a = 3 L$ и $b = 2 \delta_D$)
в моменты времени
$140\text{с}$ и $350\text{с}$,
а сверху --- пространственное распределение скорости течения раствора
во всей расчётной области.
Этот рисунок подтверждает справедливость оценок
\eqref{eq:viscous_layer} и \eqref{eq:diffusion_layer}
толщин $\delta_U$ и $\delta_D$ вязкостного и диффузионного слоёв.

\begin{figure}
  \centering
  \includegraphics[width=.5\textwidth]{pic/flat_wide_onecomp_plate_concentration_140s}%
  \includegraphics[width=.5\textwidth]{pic/flat_wide_onecomp_plate_concentration_350s}

  \includegraphics[width=.6\textwidth]{pic/flat_wide_onecomp_plate_velocity}

  \caption{\label{fig:flat_wide_plate_illustration}%
    Иллюстрация к двумерной постановке задачи об адсорбции
    на стенке бесконечно высокого канала;
    \textbf{сверху:} концентрация стрептавидина в растворе
    спустя 140с (слева) и 350с (справа) после нулевого момента времени,
    \textbf{снизу:} распределение скорости течения в объёме раствора
  }
\end{figure}

На рисунке~\ref{fig:flat_wide_onecomp_alphas}
представлено сравнение зависимости средней концентрации
связвашегося стрептаведина в данной (двумерной) постановке
от времени с аналогичными зависимостями в одномерной постановке
при значениях параметра
$\alpha \in \left\{ 1, \sqrt{2}, 2 \right\}$
(см.~\eqref{eq:viscous_layer_alpha}).
Этот рисунок говорит как о разумности применения одномерного приближения
с оценками~\eqref{eq:viscous_layer} и~\eqref{eq:diffusion_layer},
так и о разумности выбора $\alpha = \sqrt{2}$.

\begin{figure}
  \centering
  \includegraphics[width=.7\textwidth]{pic/flat_wide_onecomp_streptalphas}

  \caption{%
    \label{fig:flat_wide_onecomp_alphas}%
    Сравнение зависимости средней концентрации
    связвашегося стрептаведина в двумерной постановке
    от времени с аналогичными зависимостями в одномерной постановке
    при значениях параметра
    $\alpha \in \left\{ 1, \sqrt{2}, 2 \right\}$
  }

\end{figure}

% TODO показать влияние размера сетки на результаты

% \FloatBarrier{}
\Subsection{Адсорбция на поверхность бесконечно высокого канала с двумя видами сайтов связывания}

\subsubsection*{Одномерная постановка}

Задача во многом похожа на предыдущую, но теперь имеются 2 вида сайтов связывания:
A и B, соответствующие специфическому и неспецифическому связыванию.
Химические константы равны
$k_{A,f} = 3 \cdot 10^4 \left(\text{М}\cdot\text{с}\right)^{-1}$,
$k_{B,f} = 10^3 \left(\text{М}\cdot\text{с}\right)^{-1}$,
$K_{A,a} = 10^8 \text{М}^{-1}$, $K_{B,a} = 10^6 \text{М}^{-1}$.
Это соответствует примерно кинетике связывания белка A и белка G с
иммуноглобулином G, взятой из~\cite{bib:protein_a_b_ig_g},
с понижением $k_{B,f} = k_{G,f}$ и $K_{B,a} = K_{G,a}$ на порядок.
Коэффициент диффузии принимается равным
$D = 2 \cdot 10^{-5} \text{мм}^2/\text{с}$,
а масса молекулы аналита --- $m = 50\text{кДа}$.
Суммарная поверхностная концентрация сайтов связывания равна
$\Gamma = 4.37 \cdot 10^{-8} \text{моль}/\text{м}^2 = 0.026 \text{нм}^{-2} = 1/\left(38.5\text{нм}^2\right)$.
Доли $a$ и $b$ сайтов A и B варьируются, но всегда $a+b=1$
(количество сайтов A $\Gamma_A = a\Gamma$, сайтов B --- $\Gamma_B = b\Gamma$).
Характерная длина $L = 200\text{мкм}$,
скорость течения в глубине раствора $U = 1.8 \text{мм}/\text{с}$.
Примеси нет.

Для расчёта толщины диффузионного слоя $\delta_D$ введём новый безразмерный
параметр $\beta$ порядка единицы:
\begin{equation}
  \delta_D = \beta\sqrt[3]{\frac{D}{\nu}} \delta_U.
  \label{eq:diffusion_layer_beta}
\end{equation}
Здесь этот параметр будет полагаться равным $1/\sqrt{2}$, что будет обосновано ниже.
Заметим, что тогда произведение $\alpha\beta = 1$, т. е.
в данной задаче для расчёта толщин вязкостного и диффузионного слоёв
можно было бы с тем же успехом просто использовать формулы
\eqref{eq:viscous_layer} и \eqref{eq:diffusion_layer}
с заменой знака $\sim$ на $=$,
но в двумерном случае есть разница, о которой будет сказано далее.%, о чём ниже. % TODO ???

Начальные условия выставляются неоднородные,
см. рис.~\ref{fig:flat_wide_ramp_ic}.
Граничные условия меняются во времени: первые $1000\text{с}$
на границе фиксируется концентрация $c_0$,
следующие $7000\text{с}$ --- нулевая концентрация, что соответствует отмывке
неспецифически адсорбировавшегося вещества.

\begin{figure}
  \centering
  \includegraphics[width=.7\textwidth]{pic/wide_two_surf_ramp}
  
  \caption{%
    \label{fig:flat_wide_ramp_ic}%
    Иллюстрация к неоднородным начальным условиям в одномерной постановке
    задачи об адсорбции на поверхность канала
  }

\end{figure}

На рисунке~\ref{fig:flat_wide_two_surf_example} предствлена зависимость
поверхностной концентрации аналита, связавшегося специфически (с сайтами A),
неспецифически (с сайтами B), и их суммы при значении параметра $a = 0.5$,
т. е. при одинаковом количестве сайтов A и B.
Благодаря меньшему сродству аналита к сайтам B, чем к сайтам A, отмывка
от неспецифически адсорбировавшегося вещества возможна
(с сохранием значимой части специфически адсорбировавшегося).

\begin{figure}
  \centering
  \includegraphics[width=.7\textwidth]{pic/flat_wide_twosurf_example}

  \caption{%
    \label{fig:flat_wide_two_surf_example}%
    Задача об одномерной адсорбции на стенку бесконечно высокого канала
    при наличии неспецифической адсорбции;
    зависимость поверхностной концентрации аналита,
    связавшегося специфически, неспецифически, и суммарно;
    доля сайтов A $a = 0.5$, т. е. сайтов A и B поровну
  }

\end{figure}

На рисунке~\ref{fig:wide_two_surf_fracs_a_sum} представлены зависимости
поверхностной концентрации связавшегося специфически (с сайтами A) и
связвашегося всего (суммарно с сайтами A и B)
аналита от времени при различных долях $a$ сайтов A.
Видно, что количество специфически связавшегося аналита существенно сильнее
зависит от параметра $a$, чем общее количество связавшегося аналита,
что может быть важно, если регистрируемый прибором сигнал пропорционален
суммарному количеству связвашегося аналита или
показания прибора иначе зависят от неспецифической адсорбции аналита.
Это наблюдение говорит о возможной важности отмывки.
Количество неспецифически связавшегося аналита не показано для наглядности.
% TODO для простоты восприятия?

\begin{figure}
  \centering
  \includegraphics[width=.7\textwidth]{pic/flat_wide_twosurf_fracs_a_sum}

  \caption{%
    \label{fig:wide_two_surf_fracs_a_sum}%
    Задача об одномерной адсорбции на стенку бесконечно высокого канала
    при наличии неспецифической адсорбции;
    цветом объединены графики зависимости специфически адсорбировавшегося и
    суммарно адсорбировавшегося аналита при фиксированном параметре $a$~---
    доле сайтов специфического связывания от общего количества сайтов связывания
    (включая неспецифические)
  }

\end{figure}


\subsubsection*{Двумерная постановка}
Значения параметров те же, что и в одномерной постановке.
Сайтов A и B специфической и неспецифической адсорбции поровну ($a = b = 0.5$).
Параметр $\alpha$ из~\eqref{eq:viscous_layer_alpha} принят равным $\sqrt{2}$.
Расчётная область --- прямоугольник со сторонами $a = 3 L$ и $b = 2 \delta_U$.
Величина скорости течения, направленной вдоль стенки канала
(нижней стороны прямоугольника) вправо,
зависит от расстояния до этой стенки $y$ как
\begin{equation}
  v(y) = U \frac{\delta_U^2 - \left(y - \delta_U\right)^2}{\delta_U^2}
       = U \frac{2 y \delta_U - y^2}{\delta_U^2},
  \label{eq:flat_wide_plate_v(y)}
\end{equation}
что изображено на рисунке~\ref{fig:flat_wide_two_surf_veldistrib}.

Сайты связывания находятся на отрезке $\left[0;L\right]$,
равноудалённом от левой и правой границ расчётной области.
Изначально концентрация аналита в раствре равна нулю всюду, кроме левой границы.
Спустя 1000с концентрация аналита на левой границе зануляется
(см. рис.~\ref{fif:flat_wide_two_surf_concdistribs}).

\begin{figure}
  \centering
  \includegraphics[width=.6\textwidth]{pic/flat_wide_twosurf_veldistrib}

  \caption{%
    \label{fig:flat_wide_two_surf_veldistrib}%
    Задача о двумерной адсорбции на стенку бесконечно высокого канала
    при наличии неспецифической адсорбции;
    параболическое распределение скорости течения
  }

\end{figure}

\begin{figure}
  \centering
  \includegraphics[width=.5\textwidth]{pic/flat_wide_twosurf_11s}%
  \includegraphics[width=.5\textwidth]{pic/flat_wide_twosurf_800s}

  \includegraphics[width=.5\textwidth]{pic/flat_wide_twosurf_999s}%
  \includegraphics[width=.5\textwidth]{pic/flat_wide_twosurf_1200s}

  \caption{%
    \label{fif:flat_wide_two_surf_concdistribs}%
    Задача о двумерной адсорбции на стенку бесконечно высокого канала
    при наличии неспецифической адсорбции;
    распределение концентрации аналита по объёму раствора в разные моменты времени
  }

\end{figure}

На рисунке~\ref{fig:flat_wide_two_surf_oneplate}
приведены рядом графики зависимостей от времени
поверхностной концентрации специфически и неспецифически связавшегося аналита
для одномерной постановки и рассматриваемой двумерной.
Схожесть свидетельствует в пользу употребимости одномерной модели.

\begin{figure}
  \centering
  \includegraphics[width=.5\textwidth]{pic/flat_wide_twosurf_plane_abs}%
  \includegraphics[width=.5\textwidth]{pic/flat_wide_twosurf_onedim_abs}

  \caption{%
    \label{fig:flat_wide_two_surf_oneplate}%
    Задача об адсорбции на стенку бесконечно высокого канала
    при наличии неспецифической адсорбции;
    сравнение зависимостей поверхностных концентраций от времени
    для двумерной и одномерной постановок
  }

\end{figure}

\subsubsection*{Выбор значения параметра $\beta$}

На рисунке~\ref{fig:flat_wide_twosurf_betas} сравнены
графики зависимостей поверхностных концентраций специфически, неспецифически и
суммарно (специфически и неспецифически) связвашегося аналита
в двумерной и в одномерной постановках при значениях параметра
$\beta \in \left\{1,1/\sqrt{2},1/2\right\}$
(все прочие параметры те же, что в описании одномерной и двумерной постановок).
В случае специфического связывания и суммы специфического и неспецифического
понятно, что наилучшим значением является $\beta = 1/\sqrt{2}$,
а в случае неспецифического связывания влияние параметра $\beta$ не наблюдается.

\begin{figure}
  \centering
  \includegraphics[width=.8\textwidth]{pic/flat_wide_twosurf_sum_betas}
  \includegraphics[width=.8\textwidth]{pic/flat_wide_twosurf_a_betas}
  \includegraphics[width=.8\textwidth]{pic/flat_wide_twosurf_b_betas}

  \caption{%
    \label{fig:flat_wide_twosurf_betas}%
    Задача об адсорбции на стенку бесконечно высокого канала
    при наличии неспецифической адсорбции;
    подбор параметра $\beta \in \left\{1,1/\sqrt{2},1/2\right\}$;
    судя по двум верхним графикам, наилучшим является значение
    $\beta_{\text{opt}} = 1/\sqrt{2}$:
    синие линии, соответствующие этому значению практически идеально ложаться
    на красные --- полученные решением двумерной задачи;
    на нижнем графике 3 синие линии неразличимы
  }

\end{figure}

% \subsubsection*{Влияние верхней части расчётной области}
% TODO


\Subsection{Адсорбция на поверхность канала конечной высоты с двумя видами сайтов связывания}

Задача почти полностью аналогична предыдущей,
единственные изменения:
\begin{description}
  \item[В двумерном случае:] другая геометрия области ---
    на верхней границе находится противоположная стенка канала,
    на которой скорость течения зануляется и отсутствуют сайты связывания;
  \item[В одномерном случае:] фиксируется (вернее сказать, варьируется)
    ширина канала $H$ (высота расчётной области в двумерном случае),
    а тощина вязкостного слоя полагается равной $\delta_U = H/2$.

\end{description}
Поле скорости течения жидкости, направленной вдоль стенок канала,
задаётся формулой~\eqref{eq:flat_wide_plate_v(y)},
что изображено на рисунке~\ref{fig:flat_narrow_two_surf_veldistrib}.
Расчёт толщины диффузионного слоя происходит согласно~\eqref{eq:diffusion_layer_beta}.
Скорость течения в центре канала (ранее --- в глубине раствора)
$U = 0.1\text{мм}/\text{с}$.
Характерные распределения концентрации аналита в объёме раствора приведены
на рисунке~\ref{fig:flat_narrow_two_surf_concdistrib}.
Значения неупомянутых параметров те же, что и в предыдущей задаче.

\begin{figure}
  \centering
  \includegraphics[width=.6\textwidth]{pic/flat_narrow_plate_veldistrib}

  \caption{%
    \label{fig:flat_narrow_two_surf_veldistrib}%
    Двумерная задача об адсорбции на стенке канала конечной высоты;
    параболическое распределение скорости течения раствора
  }

\end{figure}

\begin{figure}
  \centering
  \includegraphics[width=.6\textwidth]{pic/flat_narrow_plate_20s}

  \includegraphics[width=.6\textwidth]{pic/flat_narrow_plate_50s}

  \includegraphics[width=.6\textwidth]{pic/flat_narrow_plate_100s}

  \caption{%
    \label{fig:flat_narrow_two_surf_concdistrib}%
    Двумерная задача об адсорбции на стенке канала конечной высоты;
    характерные распределения концентрации аналита в объёме раствора
  }

\end{figure}

В этой части рассматривается зависимость наилучшего значения
$\beta_\text{opt}$ параметра $\beta$ от высоты канала $H$.
На рисунке~\ref{fig:flat_narrow_betas} приведено сравнение
результатов моделирования в одномерном и двумерном случаях
для значений высоты
$H \in \left\{\delta_U^*, \delta_U^*/2, \sqrt{\delta_U^*\delta_D^*}, 2\delta_D^*\right\}$,
где $\delta_U^*$ и $\delta_D^*$
расчитаны по формулам \eqref{eq:viscous_layer_alpha} и \eqref{eq:diffusion_layer}
соответственно ($\alpha = \sqrt{2}$, в формуле~\eqref{eq:diffusion_layer} точное равенство).
Значения $\beta$ перебирались из $\left\{1, 2, 4, 8\right\}$,
но на рисунке~\ref{fig:flat_narrow_betas} приведены только результаты,
полученные с ближайшими к наилучшему значению $\beta_\text{opt}$.

\begin{figure}
  \centering
  \includegraphics[width=.33\textwidth]{pic/beta_found_6_1}%
  \includegraphics[width=.33\textwidth]{pic/beta_found_6_2}

  \includegraphics[width=.33\textwidth]{pic/beta_found_5_1}%
  \includegraphics[width=.33\textwidth]{pic/beta_found_5_2}%
  \includegraphics[width=.33\textwidth]{pic/beta_found_5_3}

  \includegraphics[width=.33\textwidth]{pic/beta_found_4_2}%
  \includegraphics[width=.33\textwidth]{pic/beta_found_4_3}%
  \includegraphics[width=.33\textwidth]{pic/beta_found_4_4}

  \includegraphics[width=.33\textwidth]{pic/beta_found_3_3}%
  \includegraphics[width=.33\textwidth]{pic/beta_found_3_4}

  \caption{%
    \label{fig:flat_narrow_betas}%
    Задача об адсорбции на поверхность канала конечной высоты;
    поиск зависимости $\beta_\text{opt}\left(H\right)$;
    сравнение результатов моделирования в одномерном и двумерном случаях
    для значений высоты
    $H \in \left\{\delta_U^*, \delta_U^*/2, \sqrt{\delta_U\delta_D}, 2\delta_D\right\}$
    при различных $\beta \in \left\{1, 2, 4, 8\right\}$,
    (изображены только ближайшие к наилучшему варианты)
  }

\end{figure}

На рисунке~\ref{fig:flat_narrow_beta_height} представлена
приблизительная зависимость $\beta\left(H\right)$ в виде точек,
полученных с помощью графиков на рис.~\ref{fig:flat_narrow_betas},
дополненных точкой $\left(H = 2\delta_U, \beta = 1/\sqrt{2}\right)$,
соответствующей графикам на рис.~\ref{fig:flat_wide_twosurf_betas},
что правомерно, т. к. участок с сайтами связывания не оказывает влияния
на концентрацию аналита в верхней половине канала
(см. рис.~\ref{fig:flat_narrow_two_surf_concdistrib}).
Точки приведены в координатах
$x = \log_{10}{\left(H/1\text{мкм}\right)}$, $y = \log_{2}{\beta}$
и не имеют размера.

Видно, что уменьшение высоты канала $H$, соответствующее уменьшению толщины
вязкостного слоя $\delta_U$ не приводит к пропорциональному уменьшению
толщины диффузионного слоя $\delta_D$, что соответствовало бы $\beta = \text{const}$.

При дальнейшем уменьшении $H$ значение $\beta_\text{opt}$ превысит $8$,
тогда уже нельзя будет говорить о том, что параметр $\beta$ имеет порядок единицы.
Это сделает использование одномерной модели сомнительным.

\begin{figure}
  \centering
  \includegraphics[width=.8\textwidth]{pic/flat_narrow_beta_height}

  \caption{%
    \label{fig:flat_narrow_beta_height}%
    Задача об адсорбции на стенке канала конечной высоты;
    приблизительная зависимость
    $\beta_\text{opt}$ --- наилучшего значения в формуле~\eqref{eq:diffusion_layer_beta}
    для одномерного приближения, от высоты канала $H$;
    точки не имеют размера
    % оценка погрешности определения коэффициентов
    % линейной регрессии $a$ и $b$ не проведена
  }

\end{figure}


% TODO невозможность уменьшить расчётную область


% \Subsection{Адсорбция на мембране}

% \FloatBarrier{}
\Subsection{Адсорбция на микросферах}

Для микросфер отдельно рассмотрены случаи, когда течение жидкости вокруг
микросфер отсутствует, и когда микросферы обтекаются жидкостью.

\subsubsection*{В отсутствие течения}

Концентрация микросфер в растворе равна $n = 1000 \text{шт}/\text{мл}$.
Исходя из этого, характерное расстояние между сферами полагается равным
$L = n^{-1/3} = 1 \text{мм}$.
Радиус сферы равен $r = 20 \text{мкм}$.

Физически, расчётная область представляет из себя
пространство между двумя концентрическими сферами с радиусами
$R_\text{in} = r$ и $R_\text{out} = L$
(рис.~\ref{fig:sphere_stationary_concentrations} сверху).
Математически, использование циллиндрической симметрии позволяет
перейти к двумерному случаю, в котором расчётная область представляет
из себя пространство между двумя концентрическими полуокружностями
с теми же радиусами $R_\text{in} = r$ и $R_\text{out} = L$
(рис.~\ref{fig:sphere_stationary_concentrations} снизу).
Строго говоря, в отсутствии течения в данной задаче имеется
сферическая симметрия, что позволяет понизить размерность до 1
(вместо понижения только до 2 в случае циллиндрической симметрии),
но COMSOL такой возможности не подразумевает.

\begin{figure}
  \centering
  \includegraphics[width=.5\textwidth]{pic/sphere_stationary_dil100_390s_3d}%
  \includegraphics[width=.5\textwidth]{pic/sphere_stationary_dil100_390s_3d_close}

  \includegraphics[width=.6\textwidth]{pic/sphere_stationary_dil100_390s_2d}
  
  \caption{%
    \label{fig:sphere_stationary_concentrations}%
    Задача об адсорбции на микросферы;
    распределение концентрации стрептавидина
    спустя 390 секунд после начала симуляции
    при начальной концентрации стрептавидина $c_0 = 1\text{мкг}/\text{мл}$:
    \textbf{снизу} изображена в циллиндрических координатах $\left(r,z\right)$,
    \textbf{сверху} --- трёхмерное изображение,
    \textbf{правое} изображение увеличено в 10 раз, сравнительно с \textbf{левым}
  }

\end{figure}

Аналитом выступает стрептавидин, кинетика та же,
что и в на странице~\pageref{streptavidin_kinetics}.
В качестве начальных условий фиксируется равномерная концентрация
стрептавидина в объёме раствора
$c_0 \in \left\{10\text{мкг}/\text{мл}, 1\text{мкг}/\text{мл}\right\}$,
на поверхности микросферы изначально все сайты связывания свободны.
Поверхностная концентрация сайтов связывания та же,
что и на странице~\pageref{Gamma0.7}.
Через границу внешней сферы радиусом $R_\text{out} = L$ потока вещества нет
(граничные условия второго рода).

На рисунке~\ref{fig:sphere_stationary_probes}
приведены графики зависимости поверхностной концентрации
свободных сайтов связывания от времени для концентраций
$c_0 = 10\text{мкг}/\text{мл}$ и $c_0 = 1\text{мкг}/\text{мл}$.
Примечательно, что пока свободных сайтов связывания достаточно много
(больше 10\% от начального количества),
зависимости близки к линейным.

% TODO зависимость времени насыщения от c_0

\begin{figure}[h]
  \centering
  \includegraphics[width=.5\textwidth]{pic/sphere_stationary_dil10_probe}%
  \includegraphics[width=.5\textwidth]{pic/sphere_stationary_dil100_probe}

  \caption{%
    \label{fig:sphere_stationary_probes}%
    Задача об адсорбции на микросферы;
    зависимости поверхностной концентрации свободных сайтов связывания
    от времени при значениях начальной концентрации стрептавидина
    $c_0 = 10\text{мкг}/\text{мл}$ \textbf{слева} и
    $c_0 = 1\text{мкг}/\text{мл}$ \textbf{справа},
    внимания заслуживает почти линейный характер зависимостей до
    насыщения поверхности микросфер стрептавидином
  }
\end{figure}

На рисунке~\ref{fig:sphere_stationary_big_small}
приведены графики зависимости поверхностной концентрации
свободных сайтов связывания от времени для концентрации
$c_0 = 1\text{мкг}/\text{мл}$ и радиусов микросфер
$r = 20\text{мкм}$ и $r = 10\text{мкм}$.
Меньшие микросферы насыщаются стрептавидином примерно в 2 раза быстрее.

\begin{figure}
  \centering
  \includegraphics[width=.7\textwidth]{pic/sphere_stationary_big_small}

  \caption{%
    \label{fig:sphere_stationary_big_small}%
    Задача об адсорбции на микросферы;
    зависимости поверхностной концентрации свободных сайтов связывания
    от времени при значении начальной концентрации стрептавидина
    $c_0 = 1\text{мкг}/\text{мл}$ и радусах микросфер
    $r = 20\text{мкм}$ и $r = 10\text{мкм}$,
    меньшие микросферы насыщаются стрептавидином примерно в 2 раза быстрее
  }

\end{figure}

% TODO истощение стрептавидина в растворе


\FloatBarrier{}
\subsubsection*{При наличии течения}

Теперь расчётная область физически представляет из себя круглый циллиндр
с радиусом $R = L$ и высотой $H = L$,
из середины которого вырезан шар радиусом $r = 20\text{мкм}$ --- микросфера
(рис.~\ref{fig:sphere_flow_concentration_velocity}).
По-прежнему используется циллиндрическая симметрия.
Вдоль оси циллиндра течёт жидкость,
сфера неподвижна и на её границе скорость течения зануляется,
на стенках циллиндра скорость не зануляется
(рис.~\ref{fig:sphere_flow_concentration_velocity} снизу).
На входном отверстии скорость течения всюду равна $U$.

\begin{figure}
  \centering
  \includegraphics[width=.5\textwidth]{pic/sphere_flow_dil100_20s_3d_concentration}%
  \includegraphics[width=.5\textwidth]{pic/sphere_flow_dil100_20s_2d_concentration}

  \includegraphics[width=.6\textwidth]{pic/sphere_flow_dil100_2d_velocity}

  \caption{%
    \label{fig:sphere_flow_concentration_velocity}%
    Задача об адсорбции на поврехность микросферы, движущейся относительно потока жидкости;
    \textbf{сверху:} распределение концентрации аналита по объёму расчётной области
    спустя $20\text{с}$ после начала симуляции,
    \textbf{снизу:} распределение скорости течения;
    средняя скорость течения $U = 0.1\text{мм}/\text{с}$
  }

\end{figure}

Начальная концентрация стрептавидина равна нулю во всей области.
Граничная концентрация стрептавидина на входном отверстии равна
$c_0 = 1\text{мкг}/\text{мл}$.

На рисунке~\ref{fig:sphere_flow_probes} приведены графики зависимости
средней поверхностной концентрации свободных сайтов связывания
на микросфере от времени для значений скорости течения жидкости
$U = 0.1\text{мм}/\text{с}$ и $U = 1\text{мм}/\text{с}$.
На этих же графиках имеются прямые, приближающие линейные участки этих зависимостей,
согласно коэффициентам наклона этих прямых, увеличение скорости в $10$ раз
привело к увеличению скорости адсорбции аналита на поверхность микросфер
в $0.038/0.015 \approx 2.5$ раза.

\begin{figure}
  \centering
  \includegraphics[width=.5\textwidth]{pic/sphere_flow_dil100_probe}%
  \includegraphics[width=.5\textwidth]{pic/sphere_flow_dil100_fast_probe}

  \caption{%
    \label{fig:sphere_flow_probes}%
    Задача об адсорбции на поврехность микросферы, движущейся относительно потока жидкости;
    зависимость поверхностной концентрации свободных сайтов связывания от времени
    при скорости течения $U = 0.1\text{мм}/\text{с}$ (\textbf{слева}) и
    $U = 1\text{мм}/\text{с}$ (\textbf{справа});
    линейные участки экстраполированы (\textbf{пунктир}),
    для них подписаны уравнения вида $y = ax + b$
  }
\end{figure}

% \Subsection{Каплегенераторы}



\Section{Материалы и методы}

В данной работе использовалось программное обеспечение
COMSOL Multiphysics\texttrademark{} версии 5.6.
Для расчёта диффузии был использован интерфейс
Transport of Diluted Spicies.
Для расчёта конвекции --- Transport of Diluted Spicies и Laminar Flow.
Для расчёта адсорбции --- Transport of Diluted Spicies и Surface Reactions.

Для расчётов была использована виртуальная машина
на 12 ядер процессора и 16 Gb оперативной памяти.

% TODO основательно перелопатить
\Section{Результаты}

В данной работе средствами COMSOL Multiphysics\texttrademark{}
были получены модели конвекции-диффузии-адсорбции
для микроканала с плоской стенкой, по которому протекает раствор с целевым маркером,
и для микросфер, находящихся в растворе с целевым маркером.

Для модели адсорбции на плоскую поверхность микроканала
были рассмотрены случаи, когда канал имеет бесконечную высоту и
профиль скорости течения определяется величиной скорости течения
в глубине раствора, и случай, когда канал имеет конечную высоту,
а профиль скорости течения определяется геометрией канала с точностью до множителя
(средней или максимальной скорости течения в канале).
Задача была рассмотрена в двух постановках: двумерной и одномерной,
где вторую можно считать приближением первой.
Для этого были использованы представления о диффузионном и вязкостном граничных слоях.
Оказалось, что уменьшение толщины вязкосного слоя, связанное с уменьшением
высоты канала не приводит к пропорциональному уменьшению диффузионного слоя.
В одномерной постановке были дополнительно рассмотрены влияние
примеси на кинетику связывания целевого маркера и
влияние неспецифического связывания на регистрируемый биосенсором сигнал.
% были смоделированы
% двумерный и одномерный случай; в одномерном случае
% были рассмотрены модели с примесью и с наличием неспецифического связывания.

Для модели адсорбции на поверхность микросфер
были рассмотрены случаи, когда отсутствует течение жидкости
относительно микросфер и когда скорость этого течения отлична от нуля.
С точки зрения физики, в первом случае --- с нулевой скоростью течения ---
геометрия расчётной области сферическая; во втором ---
с ненулевой скоростью течения --- циллиндрическая.
Для снижения вычислительных затрат была использована циллиндрическая
симметрия, что позволило производить вычисления в двумерной области
(вместо трёхмерной) в обоих случаях.
В отсутствие течения был проварьирован размер микросфер,
а при наличии течения --- скорость течения.


\Section{Выводы}
% В работе получены модели для массопереноса в канале микрофлюидного чипа
% с плоской стенкой и массопереноса из раствора на поверхность флуоресцентных микросфер.
% Полученные в данной работе модели позволяют расчитывать массоперенос
% в канале микрофлюидного чипа с плоской стенкой, а также
% массоперенос из раствора на поверхность флуоресцентных микросфер.
В работе получены модели, позволяющие рассчитать массоперенос
в канале микрофлюидного чипа с плоской стенкой и
массоперенос из раствора на поверхность флуоресцентных микросфер.

Модель адсорбции на плоскую поверхность позволяет рассчитать
влияние примеси на кинетику адсорбции целевого маркера,
а также влияние неспецифической адсорбции на форму снимаемого сигнала биосенсора.
Также возможен расчёт влияния скорости течения жидкости вдоль поверхности канала
и высоты канала на скорость адсорбции.

Модель адсорбции на поверхность микросфер позволяет рассчитывать влияние
скорости течения жидкости относительно микросфер и размера микросфер
на время насыщения поверхности микросфер.

% TODO!!!
% Использование этих моделей позволяет оценить границы диапазона таких параметров как концентрация аналита....
% не прибегая к экспериментам требующим затрат дорогостоящих реактивов.
% При хорошем совпадении экспериментальных данных и модельных расчётов можно 
% Данные модели станут основой дальнейших расчётов, связанных с решением обратных задач.
Данные модели лягут в основу дальнейших расчётов --- как тех, которые позволят
удешевить и ускорить проектирование микрофлюидных систем,
т. к. создание компьютерной модели дешевле и быстрее,
чем создание реального устройства, что существенно упрощает прототипирование;
так и расчётов, связанных с решением обратных задач, что сделает возможным
использование биосенсора для определения параметров,
характеризующих взаимодействие исследуемых веществ.

% Результаты моделирования связывания с плоской поверхностью канала говорят о том, что
% \begin{itemize}
%   \item одномерная модель диффузионного слоя применима для первичных оценочных расчётов,
%   \item влияние примеси на кинетику связывания целевого маркера может быть существенным,
%   \item необходимо проводить отмывку для удаления неспецифически связавшегося вещества.

% \end{itemize}

% В случае связывания с поверхностью микросфер ---
% \begin{itemize}
%   \item уменьшение размера микросфер приводит к уменьшению времени насыщения поверхности микросфер целевым маркером,
%   \item скорость течения раствора существенно влияет на конечную скорость связывания с поверхностью микросфер.

% \end{itemize}

% TODO можно ли делать вывод о применимости моделей, которые не применялись?
% Данные модели применимы для моделирования и проектирования микрофлюидных аналитических систем,
% основанных на специфическом связывании с модифицированной поверхностю ---
% плоской в случае канала, например, в чипе биосенсора, и выпуклой в случае микросфер.



\Section{Благодарности}

Автор этой работы выражает благодарность 
своему научному руководителю Басманову Д. В.
за чуткое и терпеливое руководство,
своему старшему коллеге Прусакову К. А. за неоценимую помощь
на всех этапах выполнения работы.


% \bibliographystyle{unsrt}
% \bibliographystyle{plaindin}
% \bibliography{bibliography,bib/springer_handbook.bib,bib/phys_chem_hydro_layers.bib}
\printbibliography{}



\end{document}
